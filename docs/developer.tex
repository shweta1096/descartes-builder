\section{Developer Documentation}

The core of the software depends on the \texttt{QtNodes} (placeholder/nodeeditor) library which handles the node, graph, connection, and UI logic. This software extends it to support directed acyclic graphs and further logical restrictions.

The backbone of the software is the execution engine which runs on Python. Kedro is a data science framework which this software will translate the graph to. Communication between the C++ software and Python is handled through \texttt{QProcess}.

The codebase is roughly divided into \texttt{app} and \texttt{docs} folders:
\begin{itemize}
    \item \texttt{docs} holds documentation for both users and developers.
    \item \texttt{app} is structured as follows:
    \begin{itemize}
        \item \texttt{external}
        \begin{itemize}
            \item \texttt{googletest}
            \item \texttt{QtNodes}
            \item \texttt{quazip}
            \item \texttt{qtutility}
        \end{itemize}
        \item \texttt{include}
        \begin{itemize}
            \item \texttt{data}
            \item \texttt{engine}
            \item \texttt{ui}
        \end{itemize}
        \item \texttt{resources}
        \item \texttt{scripts}
        \item \texttt{src}
        \begin{itemize}
            \item \texttt{data}
            \item \texttt{engine}
            \item \texttt{ui}
        \end{itemize}
        \item \texttt{tests}
        \item \texttt{build}
    \end{itemize}
\end{itemize}

In the Builder application, custom blocks are created based on the QtNodes library. These blocks are categorized as \textbf{Processor}, \textbf{Coder}, \textbf{Trainer}, and \textbf{I/O} models. Each node can have multiple input and output ports, which are classified as either \textbf{data ports} or \textbf{function ports}.

Since the application is used to create logical ML pipelines via the UI, it enforces checks on connections between nodes to ensure validity. Each data port is assigned a \textbf{unique ID (UID)}. Multiple ports can share the same UID, indicating they represent the same type of data. Different UIDs imply semantically different data types. For example, an original dataset \texttt{X} might have a UID of 2, while its PCA-transformed version might have UID 3.

All operations related to UID assignment, overriding, and consistency are handled by the \texttt{UIDManager}. The \texttt{src/data} folder includes:
\begin{itemize}
    \item \texttt{tab\_manager}: manages multiple tabs, each representing a separate pipeline.
    \item \texttt{custom\_graph}: maintains the graph state for each tab.
    \item \texttt{UIDManager}: handles type assignments and UID logic.
    \item \texttt{block\_manager}: manages unique node/port captions and other graph-level constraints.
\end{itemize}

The \texttt{src/engine} folder contains logic to interact with the Kedro execution backend. This includes generating the necessary pipeline and catalog YAML files and invoking Kedro runs.

The \texttt{src/ui} folder manages all UI components. In particular:
\begin{itemize}
    \item The \texttt{models/} subfolder defines FDF blocks (i.e., custom Qt nodes). These include:
    \begin{itemize}
        \item Port definitions (input/output)
        \item Node operations
        \item Behavior on data/function port connections
        \item UID assignment when new ports are added  
    \end{itemize}
    \item The \texttt{side\_bar\_widgets/} and other UI files handle configuration options and Qt styling.
    \item The UI allows changing implicit types of ports and assigning annotations, both of which are reflected visually in the node editor.
\end{itemize}

Additional key files include:
\begin{itemize}
    \item \texttt{log\_manager}: handles internal logging and file system logs.
\end{itemize}

\subsection*{Source Folder Responsibilities}

\begin{longtable}{|l|p{10cm}|}
\hline
\textbf{Folder} & \textbf{Purpose} \\
\hline
\texttt{src/data} & Graph, tab, and UID logic \\
\texttt{src/engine} & Kedro pipeline generation and execution \\
\texttt{src/ui} & UI widgets and FDF block definitions \\
\texttt{include/} & Header files corresponding to source files \\
\texttt{external/} & Third-party libraries (vendored) \\
\texttt{resources/} & App resources (icons, stylesheets, etc.) \\
\texttt{scripts/} & Utility scripts for build or runtime tasks \\
\texttt{tests/} & Unit and integration tests \\
\hline
\end{longtable}